\documentclass [12pt]{beamer}
\usepackage{xcolor}	
\usepackage{tikz}
\usetheme{Warsaw}
\useoutertheme{infolines}
\usepackage{ragged2e}
\begin{document}
\section*{kholase safahat 61...63}
\subsection*{mobin rajabi}	
\begin{frame}
\justifying	
In turn, our understanding of how ethics are adhered to varies with respect to privacy, confidentiality or anonymity, voluntary and informed consent, and understanding of the elements of research risk.
Anonymity refers to processes or protections that are used to hide any unique attribute (e.g., name, address, affiliate) that can be used to personally identify an item.
E-researchers need to know that other people may have access to data stored on an Internet server, and therefore, the same assurance of privacy, confidentiality and anonymity is not possible by e-researchers
\end{frame}	
\begin{frame}
\justifying	

Hackers or unauthorized people may also have access to the stored data, so they are always a threat to the secure security of the data on an Internet server.  This threat is twofold: public access and dissemination
E-researchers should advise participants that 

they can not fully guarantee that others will not have access to the information.
Another complexity and problem that e-researchers face in terms of privacy, confidentiality or autonomy is how to obtain informed consent from online participants.
\end{frame}
\begin{frame}
\justifying	
Research will focus on two main types of analysis: communication patterns and discourse analysis.  User feedback may also be collected through surveys and / or interview data.
Are courses that are being developed for implementation in the electronic software platform.  This includes conceptualizing, implementing, and evaluating tools to support and enhance collaborative learning.
The system collects the following data: (1) computer-generated application data, (2) verification copies, and (3) virtual "artifacts"
\end{frame}
\end{document}		