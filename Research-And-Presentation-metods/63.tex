{\documentclass [8pt,a4paper]{book}

\begin{document}

\begin{flushright}
	
ETHICS AND THE E-RESEARCHER    \textbf{63}	
\end{flushright}
............................................................................................
\begin{flushleft}
EXAMPLE OF AN INFORMED CONSENT FORM FOR RESEARCH PARTICIPANTS 
\end{flushleft}

Dear Participant: 
The computer conferencing software you will be using for your courses this term is an educational software system in the beta stage of development. In order to improve upon this new software, data will be collected from the courses. Research will focus on two main types of analysis: communication patterns and discourse analysis. User feedback through surveys and/or interview data may also be collected. This study is separate from your course work and the data collected will in no way be used for course grading purposes. 


Please read the text below and, if you agree, indicate your consent by signing the form and submitting a to the instructor or by submitting it online. The anher form requires you to fill in your name in place of your signature If you have any concerns or questions that you would like addressed before completing the consent form, please send an e-mail to Irmad address).


The University and those conducting this project subscribe to the ethical conduct of research and to the protection at all times of the interests, comfort, and safety of sub-jects. This form and the information it contains are given to you for your own protection and full understanding of the procedures involved. Your signature on this form or online submission of this form will signify that you have had an adequate opportunity to consider the information. and that you voluntarily agree to participate in the project. 


The objective of the research project is to develop tools to support advanced mod-els of learning and to manage, structure, and evaluate courses that are being developed to run on the e-software platform. This involves the conceptualization, implementation, and evaluation of tools to support and enhance collaborative learning. For research purposes, the system will collect the following data: (I) computer-generated usage data, (2) confer-ence transcripts, and (3) virtual "artifacts" (e.g., common file areas, Web pages). Research will focus on two main types of analysis: communication patterns and discourse analysis. User feedback through surveys and/or interview data may also be collected. The research team greatly appreciates your cooperation in our study. 


Having been asked to participate as a subject in this research project. I understand and agree to the following: 


The computer conferencing software (e-software) is an experimental system being devel-oped and will tic used for some of the activities in my course 

Institution:

course:

course Title:

Instructor(s):

\begin{itemize}
\item 
Computer-generated usage data of student use oldie c-software mill be collected.
\item  
Transcripts of user contributions will be saved and stored.
\item 
Virtual "artifacts" created by users in online courses will be saved and stored.
\item
This consent form is requesting pennission of the faculty members, teaching 2SSiS-tants, students. and industry-based participants for the use of the data described for research purposes. 

\end{itemize}
\end{document}