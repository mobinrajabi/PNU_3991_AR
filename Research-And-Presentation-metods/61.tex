{\documentclass [10pt,a4paper]{book}

\begin{document}

\begin{flushright}
	
ETHICS AND THE E-RFSEARCHER   \textbf{61}
	
\end{flushright}			
ment of law and social policy that serves the interests of their research participants and the public  They are encouraged to contribute a portion of their professional time for little or no personal advantage. 


For most types of research in physical contexts, following these principles is rel-atively straightforward. However, Net research provides a different kind of research format—one that transforms our conceptions of time, space, and physical environ-ment. In turn, our understanding of how to follow ethical principles changes with respect to maintaining privacy, confidentiality or anonymity, voluntary and informed consent, and recognizing elements of research risk. 
\begin{flushleft}
\textbf{PRIVACY, CONFIDENTIALITY, AUTONOMY, AND THE RESPECT FOR PERSONS }
\end{flushleft}
Privacy refers to the research participants' right to control access of others to them-selves. Confidentiality refers to an agreement as to how the data collected in the study will be kept private (or confidential) through controlled access to the data collected. The terms of confidentiality are usually tailored to the needs of the participants. For example. for certain participants simply changing the name may be sufficient to pro-vide confidentiality. Alternatively, it may also be necessary for the researcher to use pseudonyms for the geographical setting and type of employment to ensure confiden-tiality with other participants. Anonymity refers to processes or safeguards that are implemented to hide any unique characteristics (e.g., names, addresses, affiliated insti-tution) that could be used to personally identify a subject. New researchers often, incorrectly, exchange the terms anonymity and confidentiality. It is important to under-stand the difference between these terms with respect to research and to use them cor-rectly. The difference between anonymity and confidentiality is that with anonymity, steps are taken by the researcher to insure that the participants' identities are not revealed to the researchers (e.g., mailed surveys), whereas with confidentiality, the researcher does know the participants' identities (e.g., face-to-face focus group inter-views), but takes steps to keep their identities confidential. Respecting a participant's need for privacy, confidentiality, or autonomy is a fundamental way in which a researcher can show respect for a person—indeed, it is a requirement of ethical prac-tice among education researchers. This respect is shown most clearly by allowing the participants to share in the decision making that affects them. To make decisions accurately and knowledgeably the respondent must be informed of all the relevant details of the research and, in particular, be provided with an opportunity to refuse to participate.


These guidelines can and should he used for Net-trased research as well. How-ever, e-researchers also need to be cognizant that other people may have access to data that is kept on an Internet server and, hence, the same assurances for privacy, con-fidentiality, and anonymity cannot he provided by the e-researcher. For example, server maintenance personnel will have access to the data that resides on the server, and these individuals will need to guarantee that they will not share or distribute any data to which they may have access. The case study by Thomas (1996) cited previously 
\end{document}